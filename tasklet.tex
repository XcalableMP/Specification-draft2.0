\chapter{Directives}
\index{directive}

\section{Tasklet Constructs}

\subsection{{\tt tasklet} Construct}

\subsubsection*{Synopsis}

% The {\tt \Directive{task}} construct defines a task that is executed by
% a specified node set.

\subsubsection*{Syntax}
\Syntax{tasklet}

\begin{tabular}{ll}
\verb![F]! & \verb|!$xmp| {\tt task on} \{{\it nodes-ref} $\vert$ {\it
 template-ref}\} \\
& {\it structured-block} \\
& \verb|!$xmp| {\tt end task} \\
& \\
\verb![C]! & \verb|#pragma xmp| {\tt task on} \{{\it nodes-ref} $\vert$
     {\it template-ref}\} \\
& {\it structured-block} \\
\end{tabular}

\subsubsection*{Description}

% When a node encounters a {\tt task} construct at runtime, it executes
% the associated block (called a {\it task}) if it is included by the node
% set specified by the {\tt on} clause; otherwise it skips executing the
% block.

% %This line was inserted by Sakagami for svn test. 

% Unless a {\tt task} construct is surrounded by a {\tt \Directive{tasks}}
% construct, {\it nodes-ref} or {\it template-ref} in the {\tt on} clause
% is evaluated by the executing node set at the entry of the task;
% otherwise {\it nodes-ref} and {\it template-ref} of the {\tt task}
% construct are evaluated by the executing node set at the entry of the
% immediately surrounding {\tt tasks} construct.
% %where the evaluation
% %results must be the same in every node in the executing node set.
% %
% The current executing node set is set to that specified by the {\tt on}
% clause at the entry of the {\tt task} construct and rewound to the last
% one at the exit.

% %When {\it nodes-ref} or {\it template-ref} is evaluated, the
% %corresponding new executing node set is created conceptually.

% %The former
% %executing node set that includes the node encountering the {\tt task}
% %construct is referred to as the ``\Term{parent executing node set}'' of
% %the new executing node set.

\subsubsection*{Restrictions}

% \begin{itemize}
% \item The node set specified by {\it nodes-ref} or {\it template-ref}
%       in the {\tt on} clause must be a subset of the parent node set.
% \end{itemize}

% \subsubsection*{Example}
% \Example{task}
% \Example{end task}

% \begin{description}

% \item[Example 1]

% In XcalableMP Fortran, copies of variables {\tt a} and {\tt b} are replicated on
% nodes {\tt nd(1)} through {\tt nd(8)}. 
% A task defined by the {\tt task} construct is executed only on {\tt nd(1)} and
% defines the copies of {\tt a} and {\tt b} on a node {\tt nd(1)}. 
% The copies on nodes {\tt nd(2)} through {\tt nd(8)} are not defined.

% In XcalableMP C, copies of variables {\tt a} and {\tt b} are replicated on
% nodes {\tt nd[0]} through {\tt nd[7]}. 
% A task defined by the {\tt task} construct is executed only on {\tt nd[0]} and
% defines the copies of {\tt a} and {\tt b} on a node {\tt nd[0]}.
% The copies on nodes {\tt nd[1]} through {\tt nd[7]} are not defined.

% \hspace{\hsize}

% \begin{minipage}{0.44\hsize}
% \begin{center}
% \begin{XFexample}
% !$xmp nodes nd(8)
% !$xmp template t(100)
% !$xmp distribute t(block) onto nd

%       real a, b;

% !$xmp task on nd(1)
%       read(*,*) a
%       b = a*1.e-6
% !$xmp end task
% \end{XFexample}
% \end{center}
% \end{minipage}
% %
% \begin{minipage}{0.51\hsize}
% \begin{center}
% \begin{XCexampleR}
% #pragma xmp nodes nd[8]
% #pragma xmp template t[100]
% #pragma xmp distribute t[block] onto nd

%     float a, b;

% #pragma xmp task on nd[0]
%     {
%         scanf ("%f", &a);
%         b = a*1.e-6;
%     }
% \end{XCexampleR}
% \end{center}
% \end{minipage}

% \vspace{1cm}

% \item[Example 2]

% According to the {\tt on} clause with a template reference,
% an assignment statement in the {\tt task} construct is
% executed by the owner of the array element {\tt a(:,j)} or {\tt a[j][:]}.

% \hspace{\hsize}

% \begin{minipage}{0.44\hsize}
% \begin{center}
% \begin{XFexample}
% !$xmp nodes nd(8)
% !$xmp template t(100)
% !$xmp distribute t(block) onto nd

%       integer i,j
%       real a(200,100)
% !$xmp align a(*,j) with t(j)

%       i = ...
%       j = ...

% !$xmp task on t(j)
%       a(i,j) = 1.0
% !$xmp end task
% \end{XFexample}
% \end{center}
% \end{minipage}
% %
% \begin{minipage}{0.51\hsize}
% \begin{center}
% \begin{XCexampleR}
% #pragma xmp nodes nd[8]
% #pragma xmp template t[100]
% #pragma xmp distribute t(block) onto nd

%     int i,j;
%     float a[100][200];
% #pragma align a[j][*] with t[j]

%     i = ...;
%     j = ...;

% #pragma xmp task on t[j]
%     a[j][i] = 1.0;
% }
% \end{XCexampleR}
% \end{center}
% \end{minipage}

% \end{description}


\subsection{{\tt taskletwait} Construct}

\subsubsection*{Synopsis}

% The {\tt \Directive{tasks}} construct is used to instruct the executing
% nodes to execute the multiple tasks it surrounds in arbitrary order.

\subsubsection*{Syntax}
\Syntax{taskletwait}

\begin{tabular}{ll}
\verb![F]! & \verb|!$xmp| {\tt taskletwait} \\
& \\
\verb![C]! & \verb|#pragma xmp| {\tt taskletwait} \\
\end{tabular}

\subsubsection*{Description}

% {\tt \Directive{task}} constructs surrounded by a {\tt tasks} construct
% are executed in arbitrary order without implicit synchronization at the
% entry of each task.
% %
% As a result, if there is no overlap between the executing node sets of
% the adjacent tasks, they can be executed in parallel.

% {\it nodes-ref} or {\it template-ref} of each task immediately
% surrounded by a {\tt tasks} construct is evaluated by the executing node
% set at the entry of the {\tt tasks} construct.

% No implicit synchronization is performed at the entry and exit of the
% {\tt tasks} construct.
% %
% %implicit synchronization is performed at the exit of the {\tt tasks}
% %construct, which guarantees that all communications issued inside child
% %tasks are completed, unless a {\tt nowait} clause is specified.

% %When a {\tt nowait} clause is specified, implicit
% %synchronization is not performed at the end of the {\tt tasks}
% %construct. Without a {\tt nowait} clause, implicit synchronization is
% %performed in order to guarantee that all communications issued inside
% %child tasks are completed.

% \subsubsection*{Example}
% \Example{tasks}
% \Example{task}
% \Example{end tasks}
% \Example{end task}

% \begin{description}
%  \item[Example 1]

% 	    Three instances of subroutine {\tt task1} are concurrently
% 	    executed by node sets {\tt p(1:500)}, {\tt p(501:800)} and
% 	    {\tt p(801:1000)}, respectively.

% \hspace{\hsize}

% \begin{minipage}{0.45\hsize}
% \begin{center}
% \begin{XFexample}
%       subroutine caller
% !$xmp nodes p(1000)
% !$xmp template tp(100)
% !$xmp distribute t(block) onto p
%       real a(100,100)
% !$xmp align a(*,k) with t(k)
%       ...
% !$xmp tasks
% !$xmp  task on p(1:500)
%         call task1(a)
% !$xmp  end task
% !$xmp  task on p(501:800)
%         call task1(a)
% !$xmp  end task
% !$xmp  task on p(801:1000)
%         call task1(a)
% !$xmp  end task
% !$xmp end tasks
%       ...
%       end subroutine
% \end{XFexample}
% \end{center}
% \end{minipage}
% %
% \begin{minipage}{0.45\hsize}
% \begin{center}
% \begin{XFexampleR}
%       subroutine task1(a)
%       ...
% !$xmp nodes q(*)=*

% !$xmp nodes p(1000)
% !$xmp distribute t(block) onto p
%       real a(100,100)
% !$xmp align a(*,k) with t(k)
%       ...
%       end subroutine
% \end{XFexampleR}
% \end{center}
% \end{minipage}

% \vspace{1cm}

%  \item[Example 2]

% 	    The first node {\tt p(1)} executes the first and the second
% 	    tasks, the final {\tt node p(8)} the second and the third
% 	    tasks, and the other nodes {\tt p(2)} through {\tt p(7)}
% 	    only the second task.

% \hspace{\hsize}

% \begin{XFexample}
% !$xmp nodes p(8)
% !$xmp template t(100)
% !$xmp distribute t(block) onto p
%       real a(100)
% !$xmp align a(i) with t(i)
%       ...
% !$xmp tasks

% !$xmp task on t(1)
%       a(1) = 0.0
% !$xmp end task

% !$xmp task on t(2:99)
% !$xmp loop on t(i)
%       do i=2,99
%         a(i) = foo(i)
%       enddo
% !$xmp end task

% !$xmp task on t(100)
%       a(100) = 0.0
% !$xmp end task

% !$xmp end tasks
% \end{XFexample}

% \end{description}


\subsection{{\tt taskletbarrier} Construct}

\subsubsection*{Synopsis}

\subsubsection*{Syntax}
\Syntax{taskletwait}

\begin{tabular}{ll}
\verb![F]! & \verb|!$xmp| {\tt taskletbarrier} \\
& \\
\verb![C]! & \verb|#pragma xmp| {\tt taskletbarrier} \\
\end{tabular}

\subsubsection*{Description}

